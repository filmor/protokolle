\documentclass{protokoll}
\newcommand{\assistent}{}
\newcommand{\versuch}{$\gamma$-Spektroskopie mit Szintillations- und Halbleiterdetektoren}
\newcommand{\nummer}{P521}

\begin{document}

\section{Einleitung}
Das Ziel dieses Versuchs ist es, die $\gamma$-Spektroskopie mit Szintillations- und Ge-
Halbleiterdetektoren zu untersuchen. Die charakteristischen Eigenschaften wie Energieaufl�sung und Nachweiswahrscheinlichkeit der beiden Detektortypen werden bestimmt und mit einander verglichen. Als Anwendung des Ge-Detektors wird abschie�end eine Bodenprobe auf Spuren von Radioaktivit�t untersucht.

\section{Theoretische Grundlagen}
\subsection{Radioaktivit�t}
% zerfall, quellen, nat�rliche, zerfallsreihen, gammastrahlunng

\subsection{Wechselwirkung von $\gamma$-Strahlung mit Materie}
% photoeffekt, compton, paarbild., abh�ngigkeit des wq dieser effekte von der energie
% nd von Z des absorbers

\subsection{Szintillationsdetektor}

\subsection{Halbleiterdetektor}
\subsubsection{B�ndermodell}
%blabla

\subsection{Impulsh�henspektrum}

\subsection{Vielkanalanalysator}

\subsection{Detektorcharakteristiken}

\subsection{Vergleich der beiden Detektortypen}

\subsection{Anwendungen der Detektoren}

\subsection{Termschemata und $\gamma$-Energien der verwendeten Quellen}

\section{Versuchsaufbau, -durchf�hrung und Beobachtungen}

\section{Zusammenfassung}

\begin{appendix}
\Literatur{quellen}
\end{appendix}

\end{document}
